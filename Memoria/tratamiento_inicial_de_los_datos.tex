\chapter{Tratamiento inicial de los datos}
\section{Descripci\'on de los datos}


\begin{wrapfigure}[17]{R}{0.5\textwidth}
\includegraphics[width=0.45\textwidth]{tuit_ejemplo}
\caption{Ejemplo de tuit.}\label{fig:tuit_ejemplo}
\end{wrapfigure} 


En pantalla, un tuit relevante para nuestro proyecto podría ser el 
mostrado en la figura ad\-ya\-cente. Sin embargo, cuando descargamos el mismo tuit a través del API Search de Twitter, 
obtenemos mucha más información, en formato JSON. 

El formato JSON ({\em Javascript Object Notation})\footnote{\url{http://www.json.org/json-es.html}} 
es un formato ligero de intercambio de datos, fácilmente interpretable (por humanos y por
máquinas). Es un formato ampliamente utilizado, siendo numerosos los
lenguajes que son capaces de usar este formato (lenguajes de la familia C,Javascript, PHP, Python, etc.). 
En JSON se pueden representar dos tipos
de estructuras: un conjunto de pares (clave,valor) con la sintaxis \{clave1:valor1, clave2:valor2,\dots\}, también
denominado {\em objeto}, y un conjunto ordenado de valores con la sintaxis [valor1, valor 2,\dots] (que se denomina {\em arreglo}). 
Un valor puede ser una cadena de caracteres con comillas dobles, un número, un valor booleano o nulo, 
un objeto o un arreglo. Esta flexibilidad permite representar datos de gran complejidad.

Como veremos en el siguiente ejemplo de un tuit en formato JSON descargado a través del API Search de Twitter,
puede resultar de ayuda pensar en un JSON como en un diccionario \{clave:valor\}, donde las claves
son cadenas de texto, y el valor es algo flexible que acomoda desde una cadena de texto a un vector de objetos 
o un nuevo diccionario. En particular, la información del tuit que considerábamos más arriba 
luce de la siguiente manera:

\bigskip


\{'\_id': ObjectId('59e0cbb03842ed08188233d7'),

\quad'contributors': None,

\quad'coordinates': None,

\quad'created\_at': 'Wed Oct 11 08:13:41 +0000 2017',

\quad'entities': \{'hashtags': [\{'indices': [90, 106],'text': 'MachineLearning'\},

				  \hspace{4cm}\{'indices': [107, 114],'text': 'Python'\}],

	\hspace{2cm} 'symbols': [],

	\hspace{2cm}'urls': [\{'display\_url': 'twitter.com/i/web/status/9…',

				\hspace{3cm}'expanded\_url': 'https://twitter.com/i/web/status/918026805080182785',

				\hspace{3cm}'indices': [116, 139],

				\hspace{3cm}'url': 'https://t.co/1WuwNRzn8z'\}],

	\hspace{2cm}'user\_mentions': []\},

\quad'favorite\_count': 1,

\quad'favorited': False,

\quad'geo': None,

\quad'id': 918026805080182785,

\quad'id\_str': '918026805080182785',

\quad'in\_reply\_to\_screen\_name': None,

\quad'in\_reply\_to\_status\_id': None,

\quad'in\_reply\_to\_status\_id\_str': None,

\quad'in\_reply\_to\_user\_id': None,

\quad'in\_reply\_to\_user\_id\_str': None,

\quad'is\_quote\_status': False,

\quad'lang': 'es',

\quad'metadata': \{'iso\_language\_code': 'es',

	 \hspace{2.3cm}'result\_type': 'recent'\},

\quad'place': None,

\quad'possibly\_sensitive': False,

\quad'retweet\_count': 0,

\quad'retweeted': False,

\quad'source': '\verb|<a href="http://twitter.com"  rel="nofollow">Twitter Web Client</a>|',

\quad'text': 'Vamos preparando el siguiente libro a estudiar que al final la movida 

\hspace{1.7cm}me está gustando... \#MachineLearning \#Python… https://t.co/1WuwNRzn8z',

\quad'truncated': True,

\quad'user': \{'contributors\_enabled': False,

\hspace{1.7cm}'created\_at': 'Sat Dec 12 09:13:46 +0000 2009',

\hspace{1.7cm}'default\_profile': False,

\hspace{1.7cm}'default\_profile\_image': False,

\hspace{1.7cm}'description': 'Me gustan las camisetas y las  zapatillas // Desarrollo y '

\hspace{3cm}'Diseño para sistemas Apple //  Creador de @GetPomodoroApp ·  '

\hspace{3cm}'@GetAtentoApp · @MADatBUS ·  @GetMeteo y...',

\hspace{1.7cm}'entities': \{'description': \{'urls': []\},

\hspace{3.5cm}'url': \{'urls': [\{'display\_url': 'desappstre.com',

\hspace{4.5cm}'expanded\_url': 'http://desappstre.com',

\hspace{4.5cm}'indices': [0,23],

\hspace{4.5cm}'url': 'https://t.co/oYY42NlHrT'\}]\}\},

\hspace{1.7cm}'favourites\_count': 1512,

\hspace{1.7cm}'follow\_request\_sent': False,

\hspace{1.7cm}'followers\_count': 211,

\hspace{1.7cm}'following': False,

\hspace{1.7cm}'friends\_count': 209,

\hspace{1.7cm}'geo\_enabled': True,

\hspace{1.7cm}'has\_extended\_profile': True,

\hspace{1.7cm}'id': 96309647,

\hspace{1.7cm}'id\_str': '96309647',

\hspace{1.7cm}'is\_translation\_enabled': False,

\hspace{1.7cm}'is\_translator': False,

\hspace{1.7cm}'lang': 'es',

\hspace{1.7cm}'listed\_count': 109,

\hspace{1.7cm}'location': 'Madrid — Mundo Real™',

\hspace{1.7cm}'name': 'Adolfo ™',

\hspace{1.7cm}'notifications': False,

\hspace{1.7cm}'profile\_background\_color': '000000',

\hspace{1.7cm}'profile\_background\_image\_url': 'http://abs.twimg.com/images/themes/theme2/bg.gif',

\hspace{1.7cm}'profile\_background\_image\_url\_https': 'https://abs.twimg.com/images/themes/theme2/bg.gif',

\hspace{1.7cm}'profile\_background\_tile': False,

\hspace{1.7cm}'profile\_banner\_url': 'https://pbs.twimg.com/profile\_banners/96309647/1501577205',

\hspace{1.7cm}'profile\_image\_url': 'http://pbs.twimg.com/profile\_images/888396793024794624/O6gHh-lJ\_normal.jpg',

\hspace{1.7cm}'profile\_image\_url\_https': 'https://pbs.twimg.com/profile\_images/888396793024794624/O6gHh-lJ\_normal.jpg',

\hspace{1.7cm}'profile\_link\_color': '1B95E0',

\hspace{1.7cm}'profile\_sidebar\_border\_color': '000000',

\hspace{1.7cm}'profile\_sidebar\_fill\_color': '000000',

\hspace{1.7cm}'profile\_text\_color': '000000',

\hspace{1.7cm}'profile\_use\_background\_image': False,

\hspace{1.7cm}'protected': False,

\hspace{1.7cm}'screen\_name': 'FitoMAD',

\hspace{1.7cm}'statuses\_count': 6893,

\hspace{1.7cm}'time\_zone': 'Madrid',

\hspace{1.7cm}'translator\_type': 'none',

\hspace{1.7cm}'url': 'https://t.co/oYY42NlHrT',

\hspace{1.7cm}'utc\_offset': 7200,

\hspace{1.7cm}'verified': False\}\}

\bigskip

La descripción de cada campo de los que integran el tuit puede encontarse en la página web de Twitter para 
desarrolladores, \url{https://developer.twitter.com/en/docs/tweets/data-dictionary/overview/tweet-object}.

\section{Obtenci\'on de los datos}
Los tuits que componen nuestro corpus de datos los hemos obtenido a través del API Streaming
de Twitter, a través de una búsqueda dirigida en el API Streaming. Esta búsqueda dirigida 
se ha realizado a través de palabras clave, asociadas a la actividad de data science, concretamente:

\begin{center}
\begin{tabular}{ccccc}
\lq\lq machine learning\rq\rq  &\lq\lq machinelearning\rq\rq  &\lq\lq datamining\rq\rq  &\lq\lq data mining\rq\rq 
&\lq\lq Python\rq\rq\\
\lq\lq SQL\rq\rq & \lq\lq hadoop\rq\rq  &\lq\lq bigdata\rq\rq  &\lq\lq big data\rq\rq  &\lq\lq pentaho\rq\rq\\
\lq\lq rstats\rq\rq &\lq\lq SAS\rq\rq &\lq\lq tableau\rq\rq
\end{tabular}
\end{center}

Esta petición se define como un vector en Python, donde la coma significa \lq\lq OR\rq\rq y el espacio
dentro de las comillas significa \lq\lq AND\rq\rq, y se incluyeron dichos términos con y sin el símbolo
\#.

También hemos incluido en la búsqueda un filtro por idioma, incluyendo el parámetro 
\lq\lq languages = ["es"]\rq\rq en la llamada al API, con el objetivo de bajar solo tuits
en un idioma. Sin embargo, el efecto de este parámetro es limitado, puesto que aunque efectivamente
solo obtenemos tuits de personas adheridas al idioma español como usuarios, esto no quiere decir
que los tuits que obtengamos sean solo en español, ya que no es un filtro por el idioma del texto del tuit.
En realidad, veremos que de hecho bajamos tuits en otros idiomas (inglés, sobre todo), lo que
nos obligará a incluir esta variable en el proceso de selección de usuarios, como veremos en la sección 
\ref{sect:limpieza_de_los_datos}.

El script en el que se realiza la llamada al API de Twitter y el primer almacenamiento de los tuits
es el script llamado {\bf download\_tweets\_stream.py}. Este script importa otros
denuestro proyecto, como {\bf OpenMongoDB.py}, que gestiona la conexión a la base de datos MongoDB
en el que se almacenarán los tuits. El lanzamiento programado de la tarea se hizo con una entrada en el gestor
de Tareas Programadas del portátil, a través del archivo {\bf streaming\_upload.bat}. Todos estos
archivos se encuentran en el repositorio de GitHub descrito en la sección \ref{sect:repositorio}.


\section{Almacenamiento}
Según van produciéndose los tuits, y nuestra \lq\lq grabadora\rq\rq los va detectando, los hemos almacenado
en una base de datos de MongoDB, en local.


\section{Revisión inicial de los datos}
Una vez almacenados los tuits, realizamos un análisis exploratorio para estudiar con qué material contábamos para el desarrollo del proyecto. Este estudio se lleva a cabo en el archivo {\bf analisis\_exploratotio.py} del repositorio de
GitHub.

* Tuits originales frente a tuits retuiteados. proporcion.
ojo, hay tuits que parece que son retuits de otros (que comienzan por RT), pero luego
no tienen los campos "retweeted\_status" o "quoted\_status". Parece que esos pueden ser bots, que 
mandan mensajes automáticos. Contarlos también.

* tuits descargados a lo largo del tiempo (por día, por hora, etc.). histograma

* número de retuits por tuit (detección de eventos): for those twits that are a retwit of another one, look for the number of times that twit has been retwitted

* relación entre número de tuits descargados y número de usuarios distintos (a lo largo del tiempo, acumulado)
* número de tuits por usuario
* localización de los tuits: ¿cuántos tienen el dato disponible? Consecuencias para geolocalización de los candidatos
* hashtags: distribución.número de hashtags por tuit.


\section{Limpieza de los datos}
\label{sect:limpieza_de_los_datos}

\subsection{Detección del idioma} 
texto del tuit
\subsection{Tipo de usuario}
persona, bot, empresa análisis bio
\subsection{Naturaleza del tuit}
texto del tuit:  IT, cientifico, analista o nodatascience (diccionarios de palabras)
					/Binario, data science o no data science
