
\begin{thebibliography}{3}
\addcontentsline{toc}{chapter}{Bibliograf\'\i a}

\bibitem{proceso_seleccion1} 
Mar\'\i a Gloria Casta\~no collado, Gerardo de la Merced L\'opez Montalvo, Jos\'e Mar\'\i a Prieto Zamora,
\textit{Gu\'\i a t\'ecnica y de buenas pr\'acticas en reclutamiento y selecci\'on de personal (R\& S)}.
Documento aprobado por la Junta de Gobierno del Colegio Oficial de Psicólogos de Madrid, Febrero de 2011.
\url{http://www.copmadrid.org/webcopm/recursos/guiatecnicabuenaspracticas.pdf}

\bibitem{proceso_seleccion2} 
\textit{Selecci\'on de personal para no especialistas}.
Andaluc\'\i a Emprende, Fundaci\'on P\'ublica Andaluza. Consejer\'\i a de Econom\'\i a y Conocimiento.
\url{https://www.andaluciaemprende.es/wp-content/uploads/2015/02/guia\-seleccion-personal.pdf}

\bibitem{lopd1} 
\textit{Ley Orgánica 15/1999, de 13 de diciembre, de Protección de Datos
de Carácter Personal}.
Jefatura del Estado BOE núm. 298, de 14 de diciembre de 1999
Referencia: BOE-A-1999-23750
\url{http://www.agpd.es/portalwebAGPD/canaldocumentacion/legislacion/estatal/common/pdfs/2014/Ley_Organica_15-1999_de_13_de_diciembre_de_Proteccion_de_Datos_Consolidado.pdf}

\bibitem{tesis_mariluz} 
María Luz Congosto Martínez,
\textit{Caracterización de usuarios y propagación de mensajes en Twitter en el entorno de temas sociales}.
Tesis doctoral.

\bibitem{twitter_wikipedia} 
``Twitter". Wikipedia. \url{https://es.wikipedia.org/wiki/Twitter}.

\bibitem{kumar_et_al}Shamanth Kumar, Fred Morstatter, Huan Liu. 
{\em  Twitter Data Analytics}. Springer (2013).

\bibitem{twitter_dev_web} Twitter Developer Dpcumentation{\url https://dev.twitter.com/}

\bibitem{nltk_book}Steven Bird, Ewan Klein, Edward Loper. {\em Natural Language Processing with Python: 
Analyzing Text with the Natural Language Toolkit}. O'Reilly (2009). \url{http://www.nltk.org/book/}

\bibitem{zissman-berkling} Marc A. Zissman, Kay M.Berkling. Automatic language identification. {\em Speech Communication}, 
Volume 35, Issues 1–2, August 2001, Pg. 115-124.

\bibitem{almeida_estevez_piad} Y. Almeida-Cruz, S. Estévez-Velarde, A.  Piad-Morffis. Detección de Idioma en Twitter.
{\em GECONTEC: Revista Internacional de Gestión del Conocimiento y la Tecnología}, Vol.2 (3), 2014.
\url{https://www.upo.es/revistas/index.php/gecontec/article/view/1081/pdf_11}

\bibitem{langid} Marco Lui, Timothy Baldwin. {\tt langid.py}: An Off-the-shelf Language Identification Tool.
{\em Proceedings of the ACL 2012 System Demonstrations}, pg. 25--30, 2012.
\url{http://www.aclweb.org/anthology/P12-3005}

\bibitem{langid2} Marco Lui, Timothy Baldwin. Accurate Language Identification of Twitter Messages. 
{\em Proceedings of the 5th Workshop on Language Analysis for Social Media (LASM) @ EACL 2014}, pg. 17-–25,
(2014). \url{http://www.aclweb.org/anthology/W14-1303}

\bibitem{equilid} David Jurgens, Yulia Tsvetkov, Dan Jurafsky. Incorporating Dialectal Variability
for Socially Equitable Language Identification. 
{\em Proceedings of the 55th Annual Meeting of the Association for Computational Linguistics (Short Papers)}, 
pg. 51-–57, (2017). \url{https://doi.org/10.18653/v1/P17-2009}

\end{thebibliography}